\documentclass[12pt]{article}

\begin{document}
\title{Single Photon Emission Computed Tomography (SPECT)}
\author{Asit Khan (17111015)}
\maketitle


\section*{Introduction}
\large


Although the principles of single photon emission computed tomography (SPECT) have been well understood for many years and several centers were using SPECT clinically in the late 1960s and early 1970s, there has been a dramatic increase in the number of SPECT installations in recent years. It is now unusual to purchase a gamma camera without SPECT capability and most new cameras are dual-headed, which can offer additional advantages in SPECT. A state-of-the-art gamma camera that is well maintained should produce high-quality SPECT images consistently, and even older cameras can produce acceptable images if care is taken. SPECT is essential for imaging the brain with either cerebral blood flow agents, such as c-HMPAO, or brain receptors, such as I-FP-CIT, and for imaging myocardial perfusion with either 201Tl or the technetium-labeled agents MIBI and tetrofosmin. SPECT is also now widely used in some aspects of skeletal imaging and can be helpful in tumor imaging with, for example, I-MIBG,In octreotide, or NeoSPECT.
\par
What is the purpose of SPECT and what is its advantage over planar imaging? Planar imaging portrays a three-dimensional (3-D) distribution of radioactivity as a 2-D image with no depth information and structures at different depths are superimposed. The result is a loss of contrast in the plane of interest due to the presence of activity in overlying and underlying structures, as shown in Figure 2.1. Multiple planar views are an attempt to overcome this problem but SPECT has been developed to tackle the problem directly. SPECT also involves collecting conventional plane views of the patient from different directions but many more views are necessary, typically 64 or 128, although each view usually has fewer counts than would be acceptable in a conventional static image. From these images a set of sections through the patient can then be reconstructed mathematically. Conventionally SPECT images are viewed in three orthogonal planes – transaxial, sagittal, and coronal – as shown in Figure 2.2. Usually the transaxial images are directly obtained from SPECT data; a particular row of pixels in each image obtained with a rotating gamma camera corresponds to particular transaxial section. The other planes are derived from a stack of transaxial sections.
\par

\section*{Week 2}
\section*{ Theory of SPECT}
In this section the emphasis will be on gamma camera SPECT. The fact that these systems can also be used for conventional imaging makes them an attractive option for any nuclear medicine department. Owing to the cost and lack of flexibility of dedicated tomographic devices producing single or multiple sections with higher resolution and better sensitivity, they are likely to remain the choice of specialist centers only. Now that gamma camera SPECT is well developed commercially and relatively inexpensive, it seems certain that interest in longitudinal or limited-angle SPECT will continue to decline and so this will not be considered further. The aim here is to consider the theory of SPECT
only in sufficient detail to enable the user to understand the principles involved and to make any necessary decisions on an informed basis. A list of likely areas for decisions by the user is given in Table 2.1.









\end{document}
